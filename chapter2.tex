This chapter is intended to explore the knowledge that is available in the related literature. The chapter begins with defining the concepts and the terminology. Then, the existing knowledge is taken into consideration, to explored the related facts and to develop the background of the study. Based on the knowledge gathered here, the conceptual framework for the research study is formulated and the methodology and the structure of the study is elaborated, towards the latter part of the chapter.

\section{Terminology}

Here, it is attempted to define and clarify the related terminology which there are a few yet, they are important. There are many different terms which are alternatively in use out there hence, it is crucial to clarify each terms, if they are synonymous, related or not.

\subsection{Randomness}\label{lbl_randomness}

Randomness is defined in many different ways, taking many different aspects into consideration. 
\begin{enumerate}
    \item Cambridge English Dictionary defines randomness as 
    \begin{enumerate}
        \item  "\textit{happening, done, or chosen by chance rather than according to a plan}"\cite{web_cambridge_def_rnd}.
        \item "\textit{by chance, or without being chosen intentionally}"\cite{web_cambridge_def_rnd}.
    \end{enumerate}
    \item According to the Oxford English Dictionary, randomness is "\textit{made, done, or happening without method or conscious decision}"\cite{web_oxford_def_rnd}.
\end{enumerate}

As per the second definition provided by the Cambridge dictionary, the phrase \textit{without being chosen intentionally} emphasizes the fact that, there should not be some entity, that influences the choice made by the system. Further, the definition given by the Oxford dictionary also emphasizes on the fact that \textit{absolute lack of bias} should be a definite characteristic of randomness.

\subsection{Non-deterministic Behavior}\label{lbl_nd_behave}

Behavior of a system are said to be non-deterministic, even if everything that can be known about a system at a given time is known with all available details about the system, it is still not possible to predict the state at a future time. As per the Cambridge dictionary, "\textit{deterministic}" is an adjective, which means "\textit{Relating to the philosophical doctrine that all events, including human action, are ultimately determined by causes regarded as external to the will}"\cite{web_cambridge_def_determin}.

Some of the problems which demonstrate non-deterministic behaviors are modelled and examined in mathematics and computing related applications. According to Robert W. Floyd, a non-deterministic algorithm is "\textit{a conceptual device to simplify the design of backtracking algorithms}"\cite{art_floyd_non_determin}. A non deterministic algorithm that has f(n) levels might not return the same result on different runs. A non deterministic algorithm may never finish due to the potentially infinite size of the fixed height tree. A prime example of a non-deterministic problem is \textit{Prime Factorization or Integers} i.e. there is no algorithm that demonstrates deterministic behavior, to derive the prime factors of a given integer. Primality test is an extension of this problem. When both these problems are taken into consideration, even though we can easily predict the behavior for small inputs, the system becomes unpredictable.

\subsection{Non-deterministic nature of Randomness}

It is quite evident that the above two subsections \ref{lbl_randomness} and \ref{lbl_nd_behave} describes concepts which are going hand-in-hand. In fact, for a system to be random, one should not be able to precisely predict a future state of the system, even if everything is known about the system to the perfect detail. This leads to the fact that \textit{randomness is non-deterministic} i.e. being non-deterministic is an attribute of randomness.

\subsection{Forms of Randomness}

Randomness exists in a variety of forms. It is generally accepted that, randomness could primarily be sourced by one of the three (03) origins described below.

\begin{sectionlist}
\subsubsection{Environmental Randomness}

This form of randomness exists in the surroundings of a system. Certain phenomenon such as \textit{Brownian Motion} and \textit{Noise} in signal processing are prime examples for this form of randomness. 

\subsubsection{Randomness based on Initial Conditions}

The behavior of certain systems tend to demonstrate an extreme sensitivity to the initial conditions. For an instance we could consider \textit{Double-rod Pendulum}. This is a rod which is rigidly hinged from one end, and has another rod hinged to the other end of the first rod. The farthest edge of the system which is free to move, will form a trajectory when the system is given with an external force.

What is so fascinating about this system is that, a subtle variation of the initial conditions would cause the final trajectory to be drastically different. This behavior is known as \textit{Chaotic Behavior} in \textit{Chaos Theory}. 

\subsubsection{Randomness generated Intrinsically}
\end{sectionlist}

\subsection{Existence of Randomness}

\subsection{Random Generators}

\section{True-Random Generators}
section 2 of chapter 2

\section{Pseudo-Random Generators}

\section{Evaluation Criterion of Randomness}

Discusses about the various criterion available to quantify the quality of the randomness of each types of generators. Mainly based on the NIST randomness evaluation criterion.

\section{Environmental Randomness}

\section{Relative Randomness}

New concept. Should seek Chamath sir's consent whether to introduce this or not.

\section{Conceptual Framework}