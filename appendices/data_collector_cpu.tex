\chapter{CPU Metrics Collector Routine}\label{adx:grabber}
\begin{code}
    \begin{minted}[breaklines,tabsize=2]{py}
    #!/usr/bin/python3
    
    import os
    import sys
    import threading
    import time
    import psutil
    import json
    import cutil
    import re
    
    StartTime = None
    fileNameStr = None
    fileName = None
    fSource = None
    
    
    def grab():
    	global fileNameStr
    	global fSource
    
    	fileName = str(fileNameStr) + ".txt"
    	if fSource.closed:
    		fSource = open(fileName, "a")
    
    	if fileNameStr != str(cutil.epochHour()):
    		fSource.close()
    		print(fileName, " is closing...")
    		fileNameStr = str(cutil.epochHour())
    		fileName = fileNameStr + ".txt"
    		fSource = open(fileName, "a")
    		print("DONE! Started logging in a new file ", fileName)
    
    	metrics = dict()
    
    	# Retrieval of CPU stats
    	res = psutil.cpu_stats()
    	resType = type(res).__name__
    	data = dict(res._asdict())
    	data["temperature"] = psutil.sensors_temperatures(1)  # TODO: Fahrenheit? Or Celsius
    	metrics[resType] = data
    
    	# Retrieval of Memory Stats
    	res = psutil.virtual_memory()
    	resType = type(res).__name__
    	data = dict(res._asdict())
    	metrics[resType] = data
    
    	# Retrieval of Disk Usage Stats
    	res = psutil.disk_usage("/")
    	resType = type(res).__name__
    	data = dict(res._asdict())
    	metrics[resType] = data
    
    	fSource.write(json.dumps(metrics) + "\n")
    
    
    def getConnections():
    	for con in psutil.net_connections():
    		try:
    			yield [psutil.Process(pid=con.pid).name(), con]
    		except psutil.AccessDenied:
    			continue
    
    
    def toSeconds(timeStr):
    	if re.match(r'^[\d]+[hHmMsS]$', timeStr, re.M | re.I) is None:
    		raise ValueError("Invalid input string.")
    	timeUnit = (timeStr[-1])
    	timeAmt = int(timeStr.replace(timeUnit, ""))
    	timeUnit = timeUnit.lower()
    
    	seconds = {
    		'h': lambda timeAmt: timeAmt * 60 * 60,
    		'm': lambda timeAmt: timeAmt * 60,
    		's': lambda timeAmt: timeAmt
    	}[timeUnit](timeAmt)
    
    	return seconds
    
    
    class SetInterval:
    	def __init__(self, interval, action):
    		self.interval = interval
    		self.action = action
    		self.stopEvent = threading.Event()
    		thread = threading.Thread(target=self.__setInterval)
    		thread.start()
    
    	def __setInterval(self):
    		nextTime = time.time() + self.interval
    		while not self.stopEvent.wait(nextTime - time.time()):
    			nextTime += self.interval
    			self.action()
    
    	def cancel(self):
    		self.stopEvent.set()
    
    
    def main():
    	global StartTime
    	global fileNameStr
    	global fileName
    	global fSource
    
    	interval = int(sys.argv[1])
    	cycles = toSeconds(sys.argv[2])
    
    	StartTime = time.time()
    	fileNameStr = str(cutil.epochHour())
    	fileName = fileNameStr + ".txt"
    	fSource = open(fileName, "a")
    
    	# start action every {interval} seconds
    	inter = SetInterval(interval, grab)
    	print('Commenced reading datasources : {:.1f}s'.format(time.time()))
    
    	# will stop interval in {cycles} seconds
    	t = threading.Timer(cycles, inter.cancel)
    	t.start()
    
    
    if __name__ == '__main__':
    	try:
    		main()
    	except (KeyboardInterrupt, SystemExit):
    		print('Shutting down...')
    		try:
    			sys.exit(0)
    		except SystemExit:
    			os._exit(0)
    
    	except IndexError as e:
    		print("USAGE:python grab.py <<Interval>> <<Duration in H|M|S>>")
    \end{minted}
    \caption{Python implementation of the CPU Data capturing routine}
    \label{lst:py_class_float}
\end{code}