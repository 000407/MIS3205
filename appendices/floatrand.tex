\chapter{FloatRAND - Random Number Generator}

This is the Python implementation of the RNG based on the floating point conversion.

\begin{code}
    \begin{minted}[breaklines,tabsize=2]{py}
        import numpy as np
        from loggin import log
        
        class FloatRAND(object):
        	"""docstring for FloatRAND"""
        	def __init__(self, seed1, seed2):
        		try:
        			n = int(seed1)
        			d = int(seed2)
        			div = ('%.20f' % (n / d))
        			seed = ''.join(div.split('.'))
        			# Validity and the complexity constraints could be implemented here
        		except ValueError as e:
        			log("Invalid seed(s)! Each should consist of only numbers.")
        		super(FloatRAND, self).__init__()
        		self.seed = seed
        
        	def __arrayTimes2(self, arr):
        		c = 0
        
        		for i in reversed(range(len(arr))):
        			i2 = c + (arr[i] * 2);
        			c = i2 // 10
        			arr[i] = i2 % 10
        
        		return [c, arr]
        
        	def generate(self, length, seed = None):
        		if(seed != None):
        			self.seed = seed
        		f = [int(e.encode('ascii')) for e in self.seed]
        
        		zero = [0] * len(f)
        
        		bitStr = np.array([0] * length)
        
        		for x in range(0, length):
        			if f == zero:
        				break
        
        			[c, f] = self.__arrayTimes2(f)
        			bitStr[x] = c
        
        		return bitStr
        
        	def saveToFile(self, bitStr, fname = None):
        		if(fname == None):
        			fname = '{}.txt'.format(self.seed)
        		bitStr = map(str, bitStr);
        		file = open(fname, 'w')
        		file.write(''.join(bitStr))
        		file.close()
    \end{minted}
    \caption{Python implementation of the \texttt{FloatRAND} class - A RNG based on Floating Point Conversion}
    \label{lst:py_class_float}
\end{code}