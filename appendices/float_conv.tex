\chapter{Floating Point Converter - Python}
\begin{code}
    \begin{minted}[breaklines,tabsize=2]{py}
    class Float(object):
    	def __init__(self, l, m, n):
    		super(Float, self).__init__()
    
    		n = str(n)
    		n1 = n.split('.')
    
    		# Resolution of sign and integer
    		n = int(n1[0])
    		n = '{0:b}'.format(n)
    		n = n.split('-')
    		s = 1 if not n[0] else 0
    		n = n[1] if not n[0] else n[0]
    		n = [int(e.encode('ascii')) for e in n]
    
    		e = l - m - 1;
    
    		# Resolution of exponent
    		decPos = len(n) - 1
    
    		
    		if len(n) == 1 and n[0] == 0:
    			decPos -= 1
    
    		# Resolution of float
    		f = n1[1]
    		f = [int(e.encode('ascii')) for e in f]
    
    		zero = [0] * len(f)
    
    		sig = []
    
    		decPosFound = decPos > -1
    
    		while True:
    			if f == zero:
    				break
    			
    			[of, f] = self.__arrayTimes2(f)
    
    			if not decPosFound:
    				if of == 1:
    					decPosFound = True
    				else:
    					decPos -= 1
    			
    			if len(sig) >= (m - decPos):
    				break
    
    			sig.append(of)
    
    		n = n[1:] + sig[sig.index(1):]
    
    		n += [0] * (m - len(n))
    		
    		decPos += ((2 ** e) >> 1) - 1
    		exp = [int(e.encode('ascii')) for e in ('{0:b}'.format(decPos))]
    		
    		self.n = [s] + exp + n
    
    	def __arrayTimes2(self, arr):
    		of = 0
    		for i in reversed(range(len(arr))):
    			x = of + (arr[i] * 2)
    			of = x // 10
    			arr[i] = x % 10
    		return [of, arr]
    \end{minted}
    \caption{Python implementation of the \texttt{Float} class}
    \label{lst:py_class_float}
\end{code}