In this introductory chapter, it is focused on elaborating the overview and developing the rationale behind this study. This chapter is intended to introduce the reader to the concept of randomness, its need and the absolute purpose of this research study. Further, structure of this thesis, by each chapter is elaborated toward the latter part of this chapter.

\section{Prologue}

Human brain is known and identified to be an extraordinary organ among all of the known living beings to date. It has been identified to be capable of doing many different things and among them \textit{pattern recognition ability} is known to be one of the most prominent, and also one of the prominent obsession. This ability and the obsession has led the human being to further expand the attempts to discover knowledge with the use of patterns and correlations.

Patterns and ability to discover them, makes a system predictable. This was absolutely necessary for most cases and applications. The availability and discover-ability of patterns in different systems, has paved the way to discover vast amounts of knowledge, causing many different advancements in each discipline.  However on the contrary, this has led to some other systems to be less reliable. Certain areas in computing especially cryptography, relies on the attribute of the system being unpredictable. For the case of cryptographic systems, more the system is predictable, more vulnerable it would be. This, and some other applications requirements such as in simulations and so forth have craved for systems to be random in their behavior.

\section{Background}

\subsection{Randomness}

\subsection{True Randomness vs. Pseudo-Randomness}

\subsection{Existence of Randomness}

\section{Motivation}

\section{Objectives and Scope}

\section{Structure of the Thesis}